\documentclass[acmsmall,nonacm,screen,review]{acmart}
\newif\ifEnableExtend
%\EnableExtendtrue
\EnableExtendfalse

\usepackage[utf8]{inputenc}
\usepackage{url}
\usepackage{color}
\newcommand{\csch}[1]{{\color{red} Christian says: #1}}
\newcommand{\Is}       {:=}
\newcommand{\set}[1]{\left\{ #1\right\}}
\newcommand{\sodass}{\,:\,}
\newcommand{\setGilt}[2]{\left\{ #1\sodass #2\right\}}
\usepackage{amsmath}
\usepackage{amsthm}
\usepackage{xspace}
\usepackage{relsize}

\newtheorem{openproblem}{Open Problem}
\newcommand{\ie}{i.\,e.,\xspace}
\newcommand{\eg}{e.\,g.,\xspace}
\newcommand{\etal}{et~al.\xspace}
\newcommand{\cov}{\term{cov}\xspace}
\newcommand{\term}[1]{\textsl{#1}}
\newcommand{\Comment}[1]{\textsl{#1}}

%%%%%%%%%%%%%%%%%%%%%%%%%%%%%
\setcopyright{none}
\copyrightyear{2024}
\acmYear{2024}
\acmDOI{}
\acmPrice{}
\acmISBN{}

\title{Algorithm Engineering Seminar: }
\author{Eric Waldherr}
\email{ft278@stud.uni-heidelberg.de, Computer Science,
4231309}
\affiliation{%
  \institution{Heidelberg University}
  \streetaddress{Im Neuenheimer Feld 205}
  \city{Heidelberg}
  \state{Baden-Württemberg}
  \country{Germany}
  \postcode{69120}
}


\date{10th of July 2024}

\begin{document}

\begin{abstract}
Insert abstract here.
\end{abstract}
\maketitle

\section{Introduction}
Over the last decades graphs have become a very important data structure that are used to save data
and run algorithms on in the modern era of computer science. Companies like Google, Facebook or Amazon use graphs with millions to billions of vertices and edges to save and compute data.\\ 
This trend of graphs getting larger and larger leads to the problem that one machine alone can not run computations on this huge data sets, because they just need to much memory. This is why the graph partitioning problem became more and more present over the time as these huge graphs need to be partitioned so that algorithms can run on multiple machines at a time. For that reason every algorithm that is computed on huge data sets and every parallel algorithm on graphs faces the graph partitioning problem at some point.\\ 
In the following chapters this problem is first explained in detail in Section 2, then there is an overview of different approaches on solving the problem is presented in Section 3. Finally there is a detailed explanation of the algorithm proposed in the paper which is also the main focus of this report in Section 4. At this point it should be noted that the most information used in this report is taken from that paper as well and this report is explaining and summarizing the paper. This report closes with a summary of the most important lessons that can be learned from this paper and this report.
\section{The Graph Partitioning Problem}
This chapter focuses on explaining the graph partitioning problem and giving the most important definitions needed to understand the following chapters.\\
Input of the problem is a Graph $G = (V,E)$.\\
If $G$ is weighted, the edges $e\in E$\ have weights $c(e)\in \mathbb{R} $.
To solve the problem $G$ needs to be partitioned into $k\in \mathbb{N}$\ partitions $V_{1},...,V_{k}$ such that $V_{1}\cup...\cup V_{k} = V\ and\ V_{i}\cap V_{j} = \emptyset\ \forall i,j\in \{1,...,k\}$.\\
An important aspect is the edge cut quality, both with unweighted and weighted graphs. That means the partitioning algorithm needs to minimize 
$\sum c((u,v)) : u\in V_{i},\ v\in V_{j}\ and\ i\neq j$ \ or $\vert \{(u,v)\in E : u\in V_{i},\ v\in V_{j}\ and\ i\neq j \}\vert$ \ respectively.\\
As the partitioning is often done to prepare a graph for a parallel algorithm, these partitions need to be balanced. This means that $\forall i\in \{1,...,k\} : \vert V_{i}\vert \ < \lceil \frac{\vert V \vert }{k} \rceil $. This reduces the amount of communication needed between the different partitions, which is the amount of communication between different processing machines in the context of parallelization.
\section{Different Approaches to solve the graph partitioning problem}
The following subchapters give an overview of the most important approaches to solve the graph partitioning problem.
\subsection{Score based approach}
This approach divides the graph in k partitions at random initially.
Then, for every vertex the amount of edges inside the same partition and outside of its partition are computed. With that information it is clear whether a node should stay in its current partition or if it should be relocated into another partition. This process is repeated until no more relocations of vertices that would improve the quality of the partition can be found.\\
This approach has the problem that it gets stuck in a local optimum easily. In addition to that, there could be a beneficial swap of vertices for the edge cut quality even if the computation says otherwise.
\subsection{Multilevel approach}
This approach consists of three phases. \\
First there is the coarsening phase. This reduces the size of the graph by contracting connectivity information. All of this information needs to be saved in order to get the whole graph together again in the later stages of the algorithm. For that reason, one disadvantage of the multilevel approach is the high memory usage, which is important in the later chapters of this report.\\
Then the next phase begins, which is the initial partition step. Now the reduced graph is partitioned into k parts. After this step is done, what is left to do is to take the other vertices, that have been ignored so far, into consideration.\\
This is done in the final step, the uncoarsening phase. This step often includes heuristics and includes can include a swapping process that is similar to the swapping process of the score based approach. Now the original graph is reconstructed with the saved connectivity information. This phase ends with the whole graph partitioned into k partitions.
\subsection{Label Propagation approach}
A very detailed view of a LP approach will be given with the algorithm proposed in the paper. For that reason this chapter only describes the general LP approach in a high-level view.\\
Initially the graph is partitioned over the given machines. The LP approach consists of many iterations. In each iterations all vertices are assigned a score that describes, whether the node should be swapped into another partition or stay in the current partition. Then the vertices with an according score are relocated to another partition. This is often done until there is no more improvement to be found by using the scores.\\
The original LP approach does not guarantee a balanced partition. A balancing is often needed in modern applications making this fact a disadvantage of the original LP approach.
\section{Detailed description of a LP based algorithm}
This section focuses on the proposed algorithm of the paper. The algorithm is a LP based approach of solving the graph partitioning problem. The algorithm consists of three phases.\\
First the LP phase, which is called Quick converging LP (QCLP) in the paper. Its goal is conducting a LP phase, but only on the most important vertices to reduce the run time of the algorithm.\\
To avoid that the first phase got stuck in a local optima, the next phase, the stabilization phase, is run.\\
Finally to keep the final partition in the given vertex and edge balance constraints the last phase ensures that the constraints are fulfilled at the cost of edge cut quality.\\
In the following subchapters the initial computations of the algorithm as well as the three phases are explained in detail.
\subsection{Initial steps}
\subsection{LP process}
\subsection{Stabilization phase}
\subsection{Edge balancing phase}
\subsection{Evaluation of the algorithm}
\section{Conclusion}
\bibliographystyle{plainnat}
\bibliography{references.bib}

\end{document}
